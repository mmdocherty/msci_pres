\documentclass{beamer}
\usepackage{media9}
 	\definecolor{coolblack}{rgb}{0.0, 0.18, 0.39}
 	\definecolor{cobalt}{rgb}{0.0, 0.28, 0.67}
 	\definecolor{darkblue}{rgb}{0.0, 0.0, 0.55}
 	\definecolor{darkpink}{rgb}{0.91, 0.33, 0.5}
 	\definecolor{debianred}{rgb}{0.84, 0.04, 0.33}
%% packages
\usepackage{multimedia}
\usepackage[absolute,overlay]{textpos}
\usepackage{subfig}
\usepackage{subcaption}
\usepackage{hyperref}

%% different frame colors
\usepackage{framed,color}
\definecolor{shadecolor}{rgb}{1,0.8,0.3}
\usepackage{fancybox}
\usepackage{mdframed}
\usepackage{tikz,lipsum,lmodern}
\usepackage[most]{tcolorbox}

%% graphics like pie-chart
\usepackage{graphicx}
\usepackage{pgf-pie}
% \usepackage{fullpage}

%% to have a grid on the slides
%  \usepackage[texcoord,grid,gridunit=cm,gridcolor=red!20,subgridcolor=green!20]{eso-pic}

%%% for arrows
\usepackage{marvosym} % \MVRIGHTarrow
\usepackage{stmaryrd} % \shortrightarrow
\usepackage{textcomp} % \textrightarrow

%% Latin fonts
\usepackage[latin1]{inputenc}
\usepackage{amsmath}
\usepackage{verbatim}
\usetikzlibrary{arrows,shapes}

%% reduce spacing in the list - itemize
% \usepackage{paralist}
% \usepackage{enumitem}

%% Chemical symbols
\usepackage[version=3]{mhchem}

%% color package
% \usepackage[dvipsnames]{xcolor}

%% page settings
%% Theme
% \usetheme{Berkeley} % theme for slides
%\usetheme{Frankfurt}
\usetheme{Madrid}

%%%%%%%%%%%%%%%%%%%%%%%%%%%%%%%%%%%%%%%%%%%%%%

%% Colors
%\usecolortheme{rose} % color for slides
% \usecolortheme{default}
\usecolortheme{seahorse}
 \definecolor{languidlavender}{rgb}{0.84, 0.79, 0.87}\definecolor{lavenderblue}{rgb}{0.8, 0.8, 1.0}
%\definecolor{c1}{rgb}{0,0.7,0.6} % some green
%\definecolor{c2}{rgb}{0.9,0.9,0.9} % some gray
%% see http://www.sharelatex.com/learn/Beamer
%\setbeamercolor*{palette primary}{fg=white,bg=c1} % upper part
%\setbeamercolor*{palette secondary}{bg=c2} % left part (background)
%\setbeamercolor*{sidebar left}{fg=white,bg=c1} % left part with links
%\setbeamerfont{section number projected}{ % section numbers
%  family=\rmfamily,
%  series=\bfseries,
%  size=\normalsize
%  }
%\setbeamercolor{section number projected}{bg=c1} % color of section numbers and others (fg: Fontm, bg:Hintergrund)
%\setbeamercolor{item projected}{bg=c1}
%\setbeamercolor{itemize item}{fg=c1}
%\setbeamercolor{author in sidebar}{fg=white}
 \setbeamercolor{footlinecolor}{fg=black,bg=lavenderblue}

%%%%%%%%%%%%%%%%%%%%%%%%%%%%%%%%%%%%%%%%%%%%%%%%%%

%% Fonts
\usefonttheme{professionalfonts} % changes fonts

%%%%%%%%%%%%%%%%%%%%%%%%%%%%%%%%%%%%%%%%%%%%%
%% Foot
%  \usenavigationsymbolstemplate{} % deafult controls off
% \setbeamertemplate{footline}[frame number] % slide number at the bottom
\setbeamertemplate{footline}
{%
	\leavevmode%
	\hbox{%
	\begin{beamercolorbox}[wd=.5\paperwidth,ht=3ex,dp=1.5ex,left,leftskip=2mm]{footlinecolor}%
 		\href{}{Matthew Docherty (email: 2259886d@student.gla.ac.uk)}
%  		\href{http://www.astro.gla.ac.uk/?p=5163}{\beamergotobutton{Link}}
	\end{beamercolorbox}%
	\begin{beamercolorbox}[wd=.3\paperwidth,ht=3ex,dp=1.5ex,left,leftskip=2mm]{footlinecolor}%
		ORCID iD: \href{https://orcid.org/0000-0003-2533-4077}{000-0003-2533-4077}
		\end{beamercolorbox}%
	\begin{beamercolorbox}[wd=.2\paperwidth,ht=3ex,dp=1.5ex,right,rightskip=2mm]{footlinecolor}%
		\insertframenumber{} / \inserttotalframenumber%
	\end{beamercolorbox}%
	}%
	\vskip0pt%
}

%% new commands
\newcommand{\cm}[1]{{\tt \textcolor{orange}{#1}}}
\newcommand{\wl}[2]{\href{#2}{\textcolor{blue}{#1}}}
\newcommand{\att}[2]{\href{#2}{\textcolor{blue}{#1}}}
\newcommand{\orcid}[1]{\href{https://orcid.org/#1}{\textcolor[HTML]{A6CE39}{\aiOrcid}}}


\usebackgroundtemplate{%
\hspace{6cm}
\tikz\node[opacity=0.05,inner sep=35]{\includegraphics[height=.2\paperheight,width=.4\paperwidth]{UofG.png}};}



%% title
\author[Matthew, Chris, Hunter]{\textcolor{coolblack}{\textbf{~Matthew~M.~Docherty\inst{1}, Chris~Messenger\inst{1}, Hunter~A.~Gabbard\inst{1}}}}
\vspace{4cm}
\title[Vitamin]{\textbf{\textcolor{coolblack}{Explainable Deep-learning: Monte Carlo Methods for Gravitational Wave Inference}}}
\titlegraphic{\includegraphics[width=0.24\linewidth]{UofG} 
              \includegraphics[width=0.32\linewidth]{pres_figs/ligo2.png}
              \includegraphics[width=0.19\linewidth]{pres_figs/igr.png}
              \includegraphics[width=0.17\linewidth]{pres_figs/supa.png}}

\institute{\inst{1} SUPA, School of Physics and Astronomy, University of Glasgow, UK}
% \date{} % (count-intuitive) - comment out to INCLUDE date!
\begin{document}
{
  \usebackgroundtemplate{\includegraphics[width=\paperwidth]{White-square.png}}
  \begin{frame}
    \titlepage
    \date{}
  \end{frame}
}

\tikzstyle{every picture}+=[remember picture]
\everymath{\displaystyle}

%%%%%%%%%%%%%%%%%%%%%%%%%%%%%%%%%%%%%%%%%%%%%%%%%%%%%%%%%%%%%%%%%%%%%%%%%%%%%%%%%%%%%%%%%%%%%%%%%%%
% CONTENTS PAGE
%%%%%%%%%%%%%%%%%%%%%%%%%%%%%%%%%%%%%%%%%%%%%%%%%%%%%%%%%%%%%%%%%%%%%%%%%%%%%%%%%%%%%%%%%%%%%%%%%%%

\begin{frame}{Outline} % will auto-populate
\tableofcontents
% \tableofcontents[hideallsubsections]
\end{frame}

%%%%%%%%%%%%%%%%%%%%%%%%%%%%%%%%%%%%%%%%%%%%%%%%%%%%%%%%%%%%%%%%%%%%%%%%%%%%%%%%%%%%%%%%%%%%%%%%%%%
% SECTION 1 - Introduction
%%%%%%%%%%%%%%%%%%%%%%%%%%%%%%%%%%%%%%%%%%%%%%%%%%%%%%%%%%%%%%%%%%%%%%%%%%%%%%%%%%%%%%%%%%%%%%%%%%%

\section{Introduction \& Background}

\begin{frame}<beamer>
\frametitle{Outline}
\tableofcontents[currentsection]
\end{frame}

%%%%%%%%%%%%%%%%%%%%%%%%%%%%%%%%%%%%%%%%%%%%%%%%%%%%%%%%%%%%%%%%%%%%%%%%%%%%%%%%%%%%%%%%%%%%%%%%%%%

\begin{frame}<beamer>
\frametitle{A (Very) Brief Introduction}

\onslide<1->{\begin{block}{Parameter Estimation (PE)}
Inferring intrinsic and extrinsic parameters from their waveform signature.
\end{block}}

\onslide<2->{\begin{block}{Bayesian Statistics}
$p(x|y)\propto p(y|x)p(x).$ $\,x$ = parameters, $y$ = waveform data.
\end{block}}

\onslide<3->{\begin{block}{Sampling Algorithms}
Stochastic algorithms timescale $\mathcal{O}$(days/weeks), \texttt{Dynesty*} $\mathcal{O}$(10hrs).
\end{block}}

\onslide<4->{\begin{block}{Motivation for Deep-Learning}
Multimessenger Astronomy (MMA) benefits from fast sky localisation.
\end{block}}





\only<3->{\begin{textblock*}{55mm}(73mm,85mm)
\tiny *J. S. Speagle MNRAS Volume 493, Issue 3, April 2020
\end{textblock*}}

\end{frame}


%%%%%%%%%%%%%%%%%%%%%%%%%%%%%%%%%%%%%%%%%%%%%%%%%%%%%%%%%%%%%%%%%%%%%%%%%%%%%%%%%%%%%%%%%%%%%%%%%%%
% SECTION 2 - VItamin
%%%%%%%%%%%%%%%%%%%%%%%%%%%%%%%%%%%%%%%%%%%%%%%%%%%%%%%%%%%%%%%%%%%%%%%%%%%%%%%%%%%%%%%%%%%%%%%%%%%

\section{Our software: \texttt{VItamin}}

%%%%%%%%%%%%%%%%%%%%%%%%%%%%%%%%%%%%%%%%%%%%%%%%%%%%%%%%%%%%%%%%%%%%%%%%%%%%%%%%%%%%%%%%%%%%%%%%%%%
% SECTION 2.1 - VAE THEORY 
%%%%%%%%%%%%%%%%%%%%%%%%%%%%%%%%%%%%%%%%%%%%%%%%%%%%%%%%%%%%%%%%%%%%%%%%%%%%%%%%%%%%%%%%%%%%%%%%%%%



%%%%%%%%%%%%%%%%%%%%%%%%%%%%%%%%%%%%%%%%%%%%%%%%%%%%%%%%%%%%%%%%%%%%%%%%%%%%%%%%%%%%%%%%%%%%%%%%%%%

\begin{frame}<beamer>
\frametitle{Outline}
\tableofcontents[currentsection]
\end{frame}

\subsection{\texttt{VItamin} Structure \& Training}
%%%%%%%%%%%%%%%%%%%%%%%%%%%%%%%%%%%%%%%%%%%%%%%%%%%%%%%%%%%%%%%%%%%%%%%%%%%%%%%%%%%%%%%%%%%%%%%%%%%

\begin{frame}<beamer>
\frametitle{\texttt{VItamin} Structure \& Training}


\only<1-2>{\begin{textblock*}{30mm}(72mm,12mm)
\frame{\includegraphics[height = 70mm, width=50mm]{network_white.png}}
\end{textblock*}}

\only<1-2>{\begin{textblock*}{70mm}(70mm,85mm)
\tiny H.Gabbard \textit{et al.} arXiv preprint arXiv:1909.06296 (2019).
\end{textblock*}}

\begin{textblock*}{70mm}(0mm,30mm)
% \begin{block}{Observation 1}
\begin{description}
\onslide<1->{\item[CVAE] {\scriptsize{Conditional Variational Autoencoder}}}
\onslide<2->{\item[KL] {\scriptsize{Kullback-Leibler Divergence}}}
\onslide<2->{\item[L] {\scriptsize{Reconstruction Loss}}}
\onslide<2->{\item[H] {\scriptsize{Cost Function}}}
\end{description}
% \end{block}
\end{textblock*}

\only<2->{\begin{textblock*}{70mm}(0mm,60mm)
\centering\frame{$\,H \lesssim\frac{1}{N_\text{b}}\sum_{n=1}^{N_{\text{b}}}(L+KL)$}
\end{textblock*}}

\only<3->{\begin{textblock*}{30mm}(72mm,32mm)
\frame{\includegraphics[height = 40mm, width=50mm]{pres_figs/cost_zoom_chris_new_model.png}}
\end{textblock*}}

\end{frame}


%%%%%%%%%%%%%%%%%%%%%%%%%%%%%%%%%%%%%%%%%%%%%%%%%%%%%%%%%%%%%%%%%%%%%%%%%%%%%%%%%%%%%%%%%%%%%%%%%%%
% SECTION 2.2 - VItamin Framework 
%%%%%%%%%%%%%%%%%%%%%%%%%%%%%%%%%%%%%%%%%%%%%%%%%%%%%%%%%%%%%%%%%%%%%%%%%%%%%%%%%%%%%%%%%%%%%%%%%%%

\subsection{\texttt{VItamin} Results}

%%%%%%%%%%%%%%%%%%%%%%%%%%%%%%%%%%%%%%%%%%%%%%%%%%%%%%%%%%%%%%%%%%%%%%%%%%%%%%%%%%%%%%%%%%%%%%%%%%%

\begin{frame}<beamer>
\frametitle{\texttt{VItamin} Results}


% equation in a text block for sure!!!

\begin{textblock*}{50mm}(0mm,17mm)
\begin{itemize}
    \only<2->{\item \textbf{Plot Factfile:}
    \begin{itemize}
        \item 6 parameters
        \item 1 detector
        \item 5000 samples
    \end{itemize}}
\vspace{4mm}    
    \only<2->{\item \textbf{Choice of Parameters:}
    \begin{itemize}
        \item Less sampling time
        \item All have flat priors
        \item Phase marginalisation
    \end{itemize}}
\vspace{4mm}    
    \only<2->{\item \textbf{Result Quality:}
    \begin{itemize}
        \item Good for unimodal
        \item ``Misses" psi
        \item Incomplete training
    \end{itemize}}
\end{itemize}
\end{textblock*}

\only<2->{\begin{textblock*}{30mm}(50mm,12mm)
\frame{\includegraphics[height = 75mm, width=75mm]{pres_figs/corner_plot_chris_new_model_0_final.png}}
\end{textblock*}}

\begin{textblock*}{55mm}(67.5mm,14mm)

{\tikz[baseline]{
            \node[fill=blue!20,anchor=base] 
            {$\~p(x|y) = \int dz\,r_{\theta_1}(z|y)r_{\theta_2}(x|y,z)$};}}

\end{textblock*}

\end{frame}


%%%%%%%%%%%%%%%%%%%%%%%%%%%%%%%%%%%%%%%%%%%%%%%%%%%%%%%%%%%%%%%%%%%%%%%%%%%%%%%%%%%%%%%%%%%%%%%%%%%
% SECTION 3 - MONTE CARLO METHODS
%%%%%%%%%%%%%%%%%%%%%%%%%%%%%%%%%%%%%%%%%%%%%%%%%%%%%%%%%%%%%%%%%%%%%%%%%%%%%%%%%%%%%%%%%%%%%%%%%%%

\section{Monte Carlo Methods for Explainable Deep-learning}

%%%%%%%%%%%%%%%%%%%%%%%%%%%%%%%%%%%%%%%%%%%%%%%%%%%%%%%%%%%%%%%%%%%%%%%%%%%%%%%%%%%%%%%%%%%%%%%%%%%
% SECTION 3.1 - MONTE CARLO THEORY and practical framework
%%%%%%%%%%%%%%%%%%%%%%%%%%%%%%%%%%%%%%%%%%%%%%%%%%%%%%%%%%%%%%%%%%%%%%%%%%%%%%%%%%%%%%%%%%%%%%%%%%%
\begin{frame}<beamer>
\frametitle{Outline}
\tableofcontents[currentsection]
\end{frame}

\subsection{Monte Carlo Framework}

%%%%%%%%%%%%%%%%%%%%%%%%%%%%%%%%%%%%%%%%%%%%%%%%%%%%%%%%%%%%%%%%%%%%%%%%%%%%%%%%%%%%%%%%%%%%%%%%%%



%%%%%%%%%%%%%%%%%%%%%%%%%%%%%%%%%%%%%%%%%%%%%%%%%%%%%%%%%%%%%%%%%%%%%%%%%%%%%%%%%%%%%%%%%%%%%%%%%%

\begin{frame}
\frametitle{Monte Carlo Maths}

% $$y = 2x \onslide<2->{+ 3}$$

% include general rule, then do the colours thing again for which is which
% this is a long slide of lots of detail

\onslide<4->{\begin{itemize}
    \alert<4>{\item Monte Carlo approximation}
        \tikz[na] \node[coordinate] (n2) {};
\end{itemize}}

\onslide<2->{\begin{itemize}
    \alert<2>{\item Probability density}
        \tikz[na] \node[coordinate] (n1) {};
\end{itemize}}

\only<1->{
\begin{equation*}
I=\int dx\,f(x)\only<1>{p(x)}
\only<2->{\tikz[baseline]{
            \node[fill=red!20,anchor=base] (t1)
            {$f(x)$};
        }}
\onslide<3->{= \mathbb{E}_h(f(x))}
\onslide<4->{
\tikz[baseline]{
            \node[fill=green!20,anchor=base] (t2)
            {$\approx \frac{1}{N}\sum^{N}_{j=1}\left.f(x_j)\right|_{x_j\sim p(x)}$};
        }
}
\end{equation*}
}

\begin{tikzpicture}[overlay]
        \path[->]<2-> (n1) edge [bend left] (t1);
        \path[->]<4-> (n2) edge [bend left] (t2);
\end{tikzpicture}

\vspace{-6mm}

\onslide<7->{\begin{itemize}
    \alert<7>{\item Single posterior sample}
        \tikz[na] \node[coordinate] (n3) {};
\end{itemize}}
% \vspace{-1mm}
\onslide<5->{\begin{equation*}
\~p(x|y) = \int dz\,r_{\theta_1}(z|y)r_{\theta_2}(x|y,z)
\onslide<6->{\approx \frac{1}{N_z}\sum^{N_z}_{j=1}\left.r_{\theta_2}(
\only<6>{x}
\only<7->{\tikz[baseline]{
            \node[fill=blue!20,anchor=base] (t3)
            {$x$};
        }}
|y,z)\right|_{z_j\sim r_{\theta_1}(z|y)}}    
\end{equation*}}

\begin{tikzpicture}[overlay]
        \path[->]<7-> (n3) edge [bend left] (t3);
\end{tikzpicture}

\vspace{-8mm}
\onslide<8>{$$\~p(x|y) \sim \~p(y|x)$$}


\end{frame}

%%%%%%%%%%%%%%%%%%%%%%%%%%%%%%%%%%%%%%%%%%%%%%%%%%%%%%%%%%%%%%%%%%%%%%%%%%%%%%%%%%%%%%%%%%%%%%%%%%%
% SECTION 3.2 - MONTE CARLO INTERIM RESULTS & ONGOING WORK
%%%%%%%%%%%%%%%%%%%%%%%%%%%%%%%%%%%%%%%%%%%%%%%%%%%%%%%%%%%%%%%%%%%%%%%%%%%%%%%%%%%%%%%%%%%%%%%%%%%

\subsection{Interim Results \& Ongoing Work}

%%%%%%%%%%%%%%%%%%%%%%%%%%%%%%%%%%%%%%%%%%%%%%%%%%%%%%%%%%%%%%%%%%%%%%%%%%%%%%%%%%%%%%%%%%%%%%%%%%%


%%%%%%%%%%%%%%%%%%%%%%%%%%%%%%%%%%%%%%%%%%%%%%%%%%%%%%%%%%%%%%%%%%%%%%%%%%%%%%%%%%%%%%%%%%%%%%%%%%%

\begin{frame}<beamer>
\frametitle{Motivation \& Interim Results}


\begin{textblock*}{50mm}(0mm,17mm)
\begin{itemize}
    \only<1->{\item \textbf{Motivation:}
    \begin{itemize}
        \item Test Quality
        \item Improve Quality
        \item Importance Sampling
    \end{itemize}}
\vspace{4mm}    
    \only<2->{\item \textbf{Result Criteria:}
    \begin{itemize}
        \item Self-consistent
        \item Reproducible
    \end{itemize}}
\vspace{4mm}    
    \only<3->{\item \textbf{Plot Quality:}
    \begin{itemize}
        \item Qualatitive Success
        \item ``Misses" Inclination
    \end{itemize}}
\end{itemize}
\end{textblock*}

\only<3->{\begin{textblock*}{30mm}(50mm,11mm)
\includegraphics[height = 75mm, width=75mm]{pres_figs/self_con_white.png}
\end{textblock*}}

\end{frame}

%%%%%%%%%%%%%%%%%%%%%%%%%%%%%%%%%%%%%%%%%%%%%%%%%%%%%%%%%%%%%%%%%%%%%%%%%%%%%%%%%%%%%%%%%%%%%%%%%%%

\begin{frame}<beamer>
\frametitle{Brief Summary \& Ongoing/Future Work}

\onslide<1->{\begin{tcolorbox}[enhanced,attach boxed title to top center={yshift=-3mm,yshifttext=-1mm},
  colback=blue!5!white,colframe=blue!75!black,colbacktitle=red!80!black,
  title=Recap,fonttitle=\bfseries,
  boxed title style={size=small,colframe=red!50!black} ]
\begin{columns}[t]
\column{0.45\textwidth}
	\scriptsize  $\mathbf{1.}$~Focus on faster PE for MMA. \\
             \vspace{2mm}
             $\mathbf{3.}$~Generated \texttt{VItamin} samples and approximated their likelihood using Monte Carlo methods.
             
             
\column{0.45\textwidth}
	\scriptsize  $\mathbf{2.}$~Have trained \texttt{VItamin} to learn posterior by minimising \emph{H}. \\
             \vspace{2mm}
             $\mathbf{4.}$~Likelihoods are self-consistent from qualatitive test.
\end{columns}
\end{tcolorbox}}

\vspace{4mm}

\onslide<2->{\begin{tcolorbox}[enhanced,attach boxed title to top center={yshift=-3mm,yshifttext=-1mm},
  colback=blue!5!white,colframe=blue!75!black,colbacktitle=red!80!black,
  title=Ongoing Work,fonttitle=\bfseries,
  boxed title style={size=small,colframe=red!50!black} ]
\begin{columns}[t]
\column{0.45\textwidth}
	\scriptsize $\bullet$~Reweigh likelihoods using \texttt{Dynesty} samples via Importance Sampling. \\ 
	
             \vspace{2mm}
             $\bullet$~Rewrite code in \texttt{keras.tensorflow} for \texttt{VItamin\_c}.
\column{0.45\textwidth}
	\scriptsize  $\bullet$~Quantitative tests of likelihood reproducibility with batch size.\\
             \vspace{2mm}
             $\bullet$~Increase parameter space by accounting for non-flat priors. \\
             

\end{columns}

\end{tcolorbox}}

\end{frame}


%%%%%%%%%%%%%%%%%%%%%%%%%%%%%%%%%%%%%%%%%%%%%%%%%%%%%%%%%%%%%%%%%%%%%%%%%%%%%%%%%%%%%%%%%%%%%%%%%%%

\begin{frame}<beamer>
\frametitle{Thanks for Listening}

\begin{block}{\texttt{VItamin} Repo}
\centering\textcolor{blue}{\url{https://github.com/hagabbar/vitamin_c}}\\
Just been updated to \texttt{VItamin\_c} pre-release, go check it out!

\end{block}
% \textcolor{blue}{\url{https://github.com/hagabbar/vitamin_c}}
% Just been updated to \texttt{VItamin_c} pre-release, go check it out!!!
% Open the floor for Questions

\vspace{10mm}
\centering \large I look forwarding to answering any questions you may have!
\end{frame}


\end{document}
